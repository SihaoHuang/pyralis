%%%%%%%%%%%%%%%%%%%%%%%%%%%%%%%%%%%%%%%%%
% University/School Laboratory Report
% LaTeX Template
% Version 3.0 (4/2/13)
%
% This template has been downloaded from:
% http://www.LaTeXTemplates.com
%
% Original author:
% Linux and Unix Users Group at Virginia Tech Wiki 
% (https://vtluug.org/wiki/Example_LaTeX_chem_lab_report)
%
% License:
% CC BY-NC-SA 3.0 (http://creativecommons.org/licenses/by-nc-sa/3.0/)
%
%%%%%%%%%%%%%%%%%%%%%%%%%%%%%%%%%%%%%%%%%

%----------------------------------------------------------------------------------------
%	PACKAGES AND DOCUMENT CONFIGURATIONS
%----------------------------------------------------------------------------------------
\documentclass{article}
\usepackage{amsmath}
\usepackage{amsfonts}
\usepackage{amssymb}
\usepackage{siunitx} % Provides the \SI{}{} command for typesetting SI units
\usepackage{graphicx} % Required for the inclusion of images
\setlength\parindent{0pt} % Removes all indentation from paragraphs
\renewcommand{\labelenumi}{\alph{enumi}.} % Make numbering in the enumerate environment by letter rather than number (e.g. section 6)
\def\thesection{\Alph{section}} % make the secions be index by letters
\usepackage[section]{placeins} % make sure figures are placed in the correct section
\usepackage{pdfpages}
\usepackage[version=3]{mhchem} % chemistry typesetting
%\usepackage{times} % Uncomment to use the Times New Roman font

%----------------------------------------------------------------------------------------
%	DOCUMENT INFORMATION
%----------------------------------------------------------------------------------------
\title{ 2013-2014 Project Proposal \\ MIT Rocket Team} % Title
\author{Matt Vernacchia} % Author name
\date{ \today } % Date for the report

%----------------------------------------------------------------------------------------
% THE BODY OF THE REPORT
%---------------------------------------------------------------------------------------

\begin{document}
\maketitle
\section{Summary}
The MIT Rocket Team proposes to design, build, test, and fly a liquid bi-propellant rocket engine with an aerospike nozzle.
The production of an engine will require more rigorous engineering work, and provide better educational opportunities than the team's recent projects. We plan to fly the engine in a rocket vehicle at the Intercollegiate Rocket Engineering Competition (IREC) hosted by the Experimental Sounding Rocket Association in late June 2014. This years activities are expected to cost $13000$ to $14000$ USD.
\section{Educational Goals}
The production of an engine will provide team members with a hands-on application of a wide variety of disciplines vital to an aerospace engineering education. Applied skills will include structural analysis of engine and propellant tank pressure vessels, heat transfer, combustion thermodynamics, analysis of subsonic fluid flow through pipes, analysis of supersonic fluid flow through nozzles, engine performance predictions, launch vehicle performance predictions, sensing, controls, and experiment design.
The first seven weeks of the school year will be devoted to educating new team members. Each new member will build a rocket vehicle from a provided design. The rocket will be driven by an H-impulse class commercial solid motor, will be recovered by a pyrotechnically deployed parachute and will carry a small camera as payload. The rockets will be used for a National Association of Rocketry Level 1 certification. Each of the first six weeks will feature a lesson conducted by senior team members, followed by new members immediately applying the lesson material to the construction of their Level 1 rockets. Lesson topics will be:
\begin{enumerate}
\item Fundamentals of rocketry and simulation OpenRocket software.
\item Composite tube manufacturing.
\item CAD with Solidworks.
\item Waterjet and machine shop for fin fabrication.
\item Intro to Avionics.
\item Launch operations, motor assembly, and recovery.
\end{enumerate}

\section{Schedule}
During October, we will build a prototype of the engine. We will conduct a cold flow test on the engine to verify our engine design an modelling.
During November, we will revise the engine design based on the results of the cold flow test. We will also finalize the design of our tanks and order components for tank construction.\\
During IAP we will build a revised engine and a propellant tank. We will cold flow test, then hot fire the engine. We will also pressure-test our tank to verify the tank design and production.\\
During the spring semester we will design and build a rocket vehicle around the engine and tank. We will test subsystems of the vehicle on the ground and fly the vehicle on a commercial solid motor to verify our design. Given the development timeline of past Rocket Team projects, we expect to finish vehicle development by early May. This schedule leaves 5 weeks of overrun time before the IREC in late June.
\section{Finance}
Accounting for the prices of the components in our preliminary design, we expect the material costs for the engine, tanks, fuel and test stand to
be $1875$ USD.
\section{De-Scope Plan}
In the event of cost or time overruns in the engine development process, or insufficient funding acquisition, we will de-scope participation in the IREC and the development of a flight vehicle. In this event, our `capstone objective' for the year will be a successful static firing of the engine. As the majority of the engineering learning in the project is involved in designing the engine and associated sensing equipment, this de-scope will not seriously damage our educational goals. As our timeline is intentionally aggressive, there is a moderate risk that we will need to de-scope the project.
\end{document}
