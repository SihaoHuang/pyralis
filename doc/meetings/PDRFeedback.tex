\documentclass[12pt]{article}
\usepackage[margin=1in] {geometry}
\usepackage{fancyhdr}
\usepackage{graphicx}
\usepackage{amsmath,amsfonts,amsthm}
\usepackage{float}
\usepackage{empheq}
\usepackage{enumerate}

\title{Pyralis Preliminary Design Review} % Title
\author{Profesor Lozano\\Matt Vernacchia\\James Logan\\Connie Liu\\Kevin Sabo\\Eric Peters\\Ben Corbin }
\date{ \today } % Date for the report

%----------------------------------------------------------------------------------------
% THE BODY OF THE REPORT
%---------------------------------------------------------------------------------------

\begin{document}
\maketitle

\section*{Main Feedback \& Concerns}

\begin{itemize}
\item Threaded tungsten rod
\item Ceramic for aerospike nozzle
\item More thermodynamics analysis needed for rod \& ceramic spike
\item Aerospike's condition after impact
\item Sensors \& instrumentation for cold flow test
\end{itemize}

\section{Tungsten Rod}
Everyone who needs a body tube has one - yay! But we need to get more plywood. Look at Public Missiles LDT for parts? We have a lot of central motor mount tubes but need to check if it's enough.\\

\noindent Action items for people building rockets: Need to cut down fiberglass tubes (use bandsaw to cut 20 inches off), waterjet fins and centering rings.\\

\noindent Next week will be fin can assembly. (9/5) \\
Week after will be upper half of rocket and avionics. (9/12)\\
2 weeks after will be launch prep. (9/17)\\

\noindent We'll probably need to push launch back to \textbf{November 2, 2013.}

\section{Igniter - James}
We'll need to use a spark system as an igniter. N2O/CH4 ignites with an ignition energy of 1 J via spark at STP. We'll have an order of magnitude assumption that N20/C2H6 should take about 10 times the ignition energy of methane. Therefore, we need to provide about 10 J via a spark.

\section{Test Facilities - Matt \& James}
Talked to GTL about testing. We're good for cold flow test - fire up into a vent, but they're afraid that the heat from a hot rocket test fire will be too hot for the duct, fans, etc. However, if we can demonstrate that we can design a cooling system for the hot exhaust to cool it down so that the fans will not melt (with numbers, simulations, etc), we might be able to test in GTL blast chamber. We can test the ignition system in there.\\

\noindent Connie will follow-up with her contact at WPI.

\section{Ceramic Insulation - Jeff}
Concept = instead of having regenerative cooling, replace tubes w/ hollow cylinder of ceramic insulation. Only firing for 15 secs, so probably won't reach steady state and can model purely transient. Temp of steel will be hottest after engine fires when insulation dissipating heat, which is good since won't be taking any load (no hoop stresses, etc).\\

\noindent Jeff showed us his code for modeling the temperatures for steel and ceramics. 8 mm of insulation will keep temperature of steel to be about 600K but haven't checked if it is manufacturable, affordable, etc.\\

\noindent \textbf{Action items:}
\begin{itemize}
\item{Obasi - Find a ceramic that is affordable, manufacturable, etc.}
\item{Keep a regenerative cooling design in mind in case we want to improve engine design next year, if the ceramic idea doesn't work, make the process more efficient since we won't be loosing any energy from the combustion process, etc.}
\begin{itemize}
\item
Have solid copper 2D "cut-out shapes" as paths for regen cooling.
\end{itemize}
\end{itemize}

\noindent Should calculate (eventually) how much ethane we need to cool combustion chamber versus how much will need to go into the chamber for combustion process. Can probably do some very simple analysis for whether or not regen cooling is worth pursuing.

\section{Injector Design}
Toroidal combustion chamber and need to inject fuel and oxidizer into chamber so that they burn together and evenly. We should come up with a geometry for the injector "plates."\\

\noindent Lead: Connie; Jeff \& Kevin will step in and join if the cooling process goes well.

\section{Sensing for Cold Gas Test}
What we will definitely be measuring during cold gas test
\begin{itemize}
\item Mass flow rate (approximately 600 m/s)
\item Chamber pressure
\item Wall strain
\item Thrust
\item Chamber temperature
\item Schlieren photos - Matt talked to people in Edgerton Center; definitely practical for this application. Background-oriented Schlieren would probably be the best option for our work right now. If there are any density gradients (shockwaves), dots that the camera sees will be shifted.
\item Pressure ports down the side of the chamber? Alternative = pressure-sensitive paint (paint changes color based on what ambient pressure is like around it).
\end{itemize}

\noindent Corey will be leading Schileren photos and pressure-sensitive paint.

\section{Test Stand}
Ability to measure the thrust coming out of the engine, so the engine must be mounted on a load cell of some time. Test stand will need the ability to also cool hot exhaust if need be.\\

\noindent Connie, Hannah, resource = Jimmy from GTL, Issac

\section{Models for Aerospikes - Norman}
Norman's interested in going into more in-depth research for the modeling and general theory for aerospike engines.

\section{Fundraising - Matt \& Connie}
\begin{itemize}
\item
GEL Funding - need to talk to GEL with Henna, Matt, Connie, and Nick.
\item
Space Grant - Connie will be looking into travel grants and fellowships.
\item
Finboard - Status: Petition submitted.
\item
Edgerton - Matt \& Connie need to work on a proposal to present to Edgerton.
\item
Corporate Sponsors - After we've gone through a few successful tests, we can approach corporate sponsors for money and/or discounted supplies/equipment.
\end{itemize}

\section{Website - Norman}
BLOG ABOUT YOUR LEVEL 1 ROCKET! :) Take pictures about Level 1 rocket progress. Corey volunteered to be a team photographer. Tag the different projects.

\noindent \\To build the new dust room, we should tentatively aim to do upgrade the lab right after Level 1 certification launch and during engine design review (before we start any major fabrication and manufacturing in RT Labs). Therefore, let's aim for the beginning of November.

\section{Next Up}
\begin{itemize}
\item Pick a date for cold flow test
\item Finalize sensors list for the cold flow test
\item Explore locations for static firing
\item Throttling \& plumbing
\item Talk through rough draft of injector design \& test stand design
\end{itemize}
\end{document}